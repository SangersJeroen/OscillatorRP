\section{question 3}
\subsection{Analytical method}
We want to solve equation \ref{eq:to_solve} when a amplitude modulated external force is applied, this force follows from equation \ref{eq:force_q3}.\\

\begin{equation}
    m \ddot{x}(t)+m\frac{\omega}{Q}\dot{x}(t)+m \omega^2 x(t) = F(t)
    \label{eq:to_solve}
\end{equation}

\begin{equation}
    F(t) = F_0 t\frac{T-t}{T^2}= F_0 \frac{t}{T} - F_0 \frac{t^2}{T^2}
    \label{eq:force_q3}
\end{equation}

Since this equation \ref{eq:force_q3} is relatively simple, it can be solved using the method of undetermined coefficients as outlined in paragraph 3.5 of Boyce \cite{Boyce}.\\ The homogenous form has already been solved in the previous questions, the resulting roots are shown below in equation \ref{eq:solution_roots} and the solution in \ref{eq:solution_y_with_roots}.\\

\begin{equation}
    r_1 = -1/2 \biggl( \frac{\omega}{Q}+\sqrt{\frac{\omega^2}{Q^2}-4 \omega^2} \biggr), r_2 = 1/2 \biggl( -\frac{\omega}{Q}+\sqrt{\frac{\omega^2}{Q^2}-4 \omega^2} \biggr)
    \label{eq:solution_roots}
\end{equation}

\begin{equation}
    y_1(t) = e^{r_1 t},  y_2(t) = e^{r_2 t}
    \label{eq:solution_y_with_roots}
\end{equation}

Now for the particular solution $y_p(t)$ we use the aforementioned method, the derivation is show below. We start by assuming that $y_p(t)$ is of the shape:\\

\begin{equation*}
    y_p(t) = c_1 + c_2 \cdot t + c_3 \cdot t^2
\end{equation*}

If we then differentiate $y_p(t)$ two times and substitute the result into equation \ref{eq:to_solve} we get the following:

\begin{equation*}
    y_p'(t) = c_2 + 2c_3 \cdot t,    y_p"(t) = 2c_3
\end{equation*}

\begin{align*}
    m y"(t)+ \frac{m \omega}{Q} y'(t) + m \omega^2 y(t) &= F_0 \frac{t}{T} -F_0 \frac{t^2}{T^2}\\
    m (2a_3) + \frac{m \omega}{Q} (a_2 +2a_3 t) + m \omega^2 (a_1 + a_2 t + a_3t^2) &= F_0 \frac{t}{T} -F_0 \frac{t^2}{T^2}
\end{align*}

If we then equate the terms in front of the functions and it's derivatives we get the following:\\
\begin{align*}
    m\omega^2 a_1 + \frac{m \omega a_2}{Q} + 2ma_3 &= 0\\
    m \omega^2 a_2 + \frac{2m\omega a_3}{Q} &= \frac{F_0}{T}\\
    m \omega^2 a_3 &= -\frac{F_0}{T^2}
\end{align*}

Solving the system of equations and substituting back into $y_p(t)$ yields:

\begin{align*}
    a_1 &= \frac{F_0 ( 2Q^2 -Q T\omega -2)}{mQ^2 T^2 \omega^4}\\
    a_2 &= \frac{F_0 ( QT\omega +2)}{mQT^2\omega^3}\\
    a_3 &= \frac{-F_0}{mT^2 \omega^2}
\end{align*}

\begin{equation}
    y_p(t)=\frac{F_0 t}{mT\omega^2}(1-t/T)+\frac{F_0}{mQT\omega^3}\biggl[\frac{2(t+Q)}{T}-\frac{2}{QT\omega}-1\biggr]
\end{equation}
\clearpage
\subsection{Numerical method}
To solve equation \ref{eq:to_solve} numerically we first have to split the second order differential equations into a system of two first order equations. We do this by substituting two new time dependant functions for $y$, namely $u(t)$ and $v(t)$ equal to $y(t)$ and $y'(t)$ respectively. We can then derive the following system:\\

\begin{equation*}
    \left\{ \begin{matrix} \mbox{u(t) = y(t)}\\
    \mbox{v(t) = y'(t)} \end{matrix} \right.
\end{equation*}

So that their derivatives become:\\

\begin{equation*}
    \left\{ \begin{matrix} \mbox{u'(t) = y'(t) = v(t)}\\
    \mbox{v'(t) = y"(t)} \end{matrix} \right.
\end{equation*}

If we then substitute in these equations into the second order differential equation we get the following system:\\

\begin{align*}
    u'(t) &= y'(t) = v(t)\\
    v'(t) &= y''(t)\\
    F(t) &= m y''(t)+m\frac{\omega}{Q}y'(t)+m \omega^2 y(t)\\
\end{align*}

\begin{align*}
    v'(t) &= y''(t) = 1/m \cdot F(t) - \omega/Q\cdot y'(t)-\omega^2\cdot y(t)\\
    v'(t) &= 1/m\cdot F(t) - \omega/Q\cdot v(t)-\omega^2\cdot u(t)
\end{align*}

So that we now have the following system of of first-order linear differential equations:\\

\begin{align}
    v'(t) &= 1/m\cdot F(t)-\omega/Q\cdot v(t)- \omega^2\cdot u(t)\\
    u'(t) &= v(t)
\end{align}

This system can be easily solved numerically using the following python code:\\


\lstinputlisting[language=Python]{../Python/ode_solver.py}