\section{Appendix}

\subsection{Infinite excitation}

\subsubsection{Solving $Y(t)$}
In order to solve the particular solution to the differential equation we assume $Y(t)$ is of the shape:

\begin{equation*}
	Y(t) = a_1 cos(\omega t) + a_2 sin(\omega t)
\end{equation*}

Differentiating with respect to $t$ once and twice yields:

\begin{equation*}
	\dot{Y}(t) = - a_1 \omega \: sin(\omega t) + a_2 \omega \: cos(\omega t)
\end{equation*}
\begin{equation*}
	\ddot{Y}(t) = - a_1 \omega^2 \: cos(\omega t) - a_2 \omega^2 \: sin(\omega t)
\end{equation*}


If we substitute this for $x(t)$ in equation \ref{eq_diff} and with $F(t) = F_0 \: cos(\omega t)$ we find:

\begin{align*}
	F(t)m  &= \ddot{x}(t) + m \frac{\omega}{Q} \dot{x}(t) + m \omega^2 x(t)  \\
	F_0 \:cos(\omega t) &= m \left[ - a_1 \omega^2 \:cos(\omega t) - a_2 \omega^2 \:sin(\omega t) \right] + m \frac{\omega}{Q} \left[- a_1 \omega \:sin(\omega t) + a_2 \omega \:cos(\omega t) \right] + m \omega^2 \left[ a_1 \:cos(\omega t) + a_2 \:sin(\omega t)\right] \\
	F_0 \:cos(\omega t) &= omega^2 m \:cos(\omega t) \left(a_1 + \frac{a_2}{Q} - a_1 \right) + \omega^2 m \:sin(\omega t) \left( a_2 - \frac{a_1}{Q} - a_2 \right) \\
	F_0 \:cos(\omega t) &= a_2 \frac{\omega^2 m}{Q} \:cos(\omega t) - a_1 \frac{\omega^2 m}{Q} \:sin( \omega t) \\
\end{align*}

From this follows:

\begin{align*}
	- a_1 \frac{\omega^2 m}{Q} = 0 &  &  a_2 \frac{\omega^2 m}{Q} = F_0 \\
	a_1 = 0 &  & a_2 = \frac{F_0 Q}{\omega^2 m}
\end{align*}

Therefore the solution to differential equation \ref{eq_diff} is:

\begin{equation*}
	x(t) = c_1 e^{r_1 t} + c_2 e^{r_2 t} + \frac{F_0 Q}{\omega^2 m}sin(\omega t)
\end{equation*}

We can find $c_1$ and $c_2$ by imposing the initial conditions on the latter equation:

\begin{align*}
	x(0) = x_0 = c_1 + c_2 \\
	c_1 = x_0 - c_2
\end{align*}

\begin{align*}
	\dot{x}(t) = c_1 r_1 e^{r_1 t} + c_2 r_2 e^{r_2 t} + \frac{F_0 Q}{\omega m} cos(\omega t) \\
	\dot{x}(0) = \dot{x}_0 = c_1 r_1 + c_2 r_2 + \frac{F_0 Q}{\omega m} \\
	\dot{x}_0 = [x_0-c_2] r_1 + c_2 r_2 + \frac{F_0 Q}{\omega m} \\
	c_2(r_2 - r_1) = \dot{x}_0 - \frac{F_0 Q}{\omega m} -x_0 r_1 \\
	c_2 = \frac{1}{r_2-r_1} \left( \dot{x}_0 - \frac{F_0 Q}{\omega m} - x_0 r_1 \right)
\end{align*}

We substitute this back into the equation for $c_1$:
\begin{align*}
	c_1 &= x_0 - c_2 \\
	&= x_0 - \frac{1}{r_2-r_1} \left( \dot{x}_0 - \frac{F_0 Q}{\omega m} - x_0 r_1 \right) \\
	&= \frac{1}{r_2-r_1} \left( -\dot{x}_0 + \frac{F_0 Q}{\omega m} + x_0  r_2  \right)
\end{align*}

\subsection{Amplitude modulation}

\lstset{label=PythonCode}
\lstinputlisting[language=Python]{../Python/ode_solver.py}
\clearpage
\subsection{Analytical method Question 3}
We want to solve equation \ref{eq:to_solve_app} when a amplitude modulated external force is applied, this force follows from equation \ref{eq:force_q3_app}.\\

\begin{equation}
    m \ddot{x}(t)+m\frac{\omega}{Q}\dot{x}(t)+m \omega^2 x(t) = F(t)
    \label{eq:to_solve_app}
\end{equation}

\begin{equation}
    F(t) = F_0 t\frac{T-t}{T^2}= F_0 \frac{t}{T} - F_0 \frac{t^2}{T^2}
    \label{eq:force_q3_app}
\end{equation}

Since this equation \ref{eq:force_q3} is relatively simple, it can be solved using the method of undetermined coefficients as outlined in paragraph 3.5 of Boyce \cite{Boyce}.\\ The homogenous form has already been solved in the previous questions, the resulting roots are shown below in equation \ref{eq:solution_roots} and the solution in \ref{eq:solution_y_with_roots}.\\

\begin{equation}
    r_1 = -1/2 \biggl( \frac{\omega}{Q}+\sqrt{\frac{\omega^2}{Q^2}-4 \omega^2} \biggr), r_2 = 1/2 \biggl( -\frac{\omega}{Q}+\sqrt{\frac{\omega^2}{Q^2}-4 \omega^2} \biggr)
    \label{eq:solution_roots}
\end{equation}

\begin{equation}
    y_1(t) = e^{r_1 t},  y_2(t) = e^{r_2 t}
    \label{eq:solution_y_with_roots}
\end{equation}

Now for the particular solution $y_p(t)$ we use the aforementioned method, the derivation is show below. We start by assuming that $y_p(t)$ is of the shape:\\

\begin{equation*}
    y_p(t) = c_1 + c_2 \cdot t + c_3 \cdot t^2
\end{equation*}

If we then differentiate $y_p(t)$ two times and substitute the result into equation \ref{eq:to_solve} we get the following:

\begin{equation*}
    y_p'(t) = c_2 + 2c_3 \cdot t,    y_p"(t) = 2c_3
\end{equation*}

\begin{align*}
    m y"(t)+ \frac{m \omega}{Q} y'(t) + m \omega^2 y(t) &= F_0 \frac{t}{T} -F_0 \frac{t^2}{T^2}\\
    m (2a_3) + \frac{m \omega}{Q} (a_2 +2a_3 t) + m \omega^2 (a_1 + a_2 t + a_3t^2) &= F_0 \frac{t}{T} -F_0 \frac{t^2}{T^2}
\end{align*}

If we then equate the terms in front of the functions and it's derivatives we get the following:\\
\begin{align*}
    m\omega^2 a_1 + \frac{m \omega a_2}{Q} + 2ma_3 &= 0\\
    m \omega^2 a_2 + \frac{2m\omega a_3}{Q} &= \frac{F_0}{T}\\
    m \omega^2 a_3 &= -\frac{F_0}{T^2}
\end{align*}

Solving the system of equations and substituting back into $y_p(t)$ yields:

\begin{align*}
    a_1 &= \frac{F_0 ( 2Q^2 -Q T\omega -2)}{mQ^2 T^2 \omega^4}\\
    a_2 &= \frac{F_0 ( QT\omega +2)}{mQT^2\omega^3}\\
    a_3 &= \frac{-F_0}{mT^2 \omega^2}
\end{align*}

\begin{equation}
    y_p(t)=\frac{F_0 t}{mT\omega^2}(1-t/T)+\frac{F_0}{mQT\omega^3}\biggl[\frac{2(t+Q)}{T}-\frac{2}{QT\omega}-1\biggr]
\end{equation}
\clearpage






