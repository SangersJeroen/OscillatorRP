\section{conclusion}
After having worked through all the question we have learned a new method for solving differential equations, we have gained new intuition on how different systems react to different damping and driving forces. In question 3 we realised that it is easier and faster to solve differential equations numerically than analytically.
More in depth, in the first question we learned that for an infinitely high and a negative $Q$-factor the amplitude of the oscillations keeps growing while, for the first scenario it's the case that their is no friction but in the second scenario it's the case that the friction force actually drives the system. For a negative $Q$-factor the amplitude grows much faster than for a frictionless system.
In the second question we looked at finite driving force of limited timespan. Where we noticed that the $Q$-factor has different effects, due to the finite driving period a system with friction now decays to its equilibrium position over time.
For the third and final question we looked at a finite driving period of a force that has a different period. We noticed that a small driving period looks and works like an impulse, it looks like the oscillation started with a initial velocity instead of at rest.