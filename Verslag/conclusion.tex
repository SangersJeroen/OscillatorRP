\section{conclusion}
After having worked through all the question we have learned a new method for solving differential equations, we have gained new intuition on how different systems react to different damping and driving forces. In question 3 we realised that it is easier and faster to solve differential equations numerically than analytically.
More in depth, in the case of infinite excitation, we learned that for an infinitely high and a negative $Q$-factor the amplitude of the oscillations keeps growing while, for the first scenario it's the case that their is no friction but in the second scenario it's the case that the friction force actually drives the system. For a negative $Q$-factor the amplitude grows much faster than for a frictionless system.
For the situation with the finite excitation time, we noticed that the $Q$-factor has different effects. The system with friction decays to its equilibrium position over time, the frictionless system will reach a steady state and the system with 'inverse' friction keeps amplifying resulting in little difference between the different excitation times.
For the third situation we looked at a finite amplitude modulation. We noticed that a small driving period looks and works like an impulse, it looks like the oscillation started with a initial velocity instead of at rest. What's more, this type of modulation can alter the movement of the oscillation such that it better resembles only one broad pulse. 