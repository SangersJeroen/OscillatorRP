\section{question 1}

Equation \ref{eq_diff} is a second order linear differential equation. According to theorem 3.6.1 in Boyce \cite{Boyce}:

\begin{equation}
	x(t) = c_1 y_1(t) + c_2 y_2(t) + Y(t)
\end{equation}

\begin{equation}
	\label{eq_particular}
	Y(t) = - y_1(t)\int_{t0}^t \frac{y_2(s) F(s)}{W[y_1,y_2](s)}ds + y_2(t)\int_{t0}^t \frac{y_1(s) F(s)}{W[y_1,y_2](s)}ds
\end{equation}

With $W[y_1,y_2]$, the Wronskian as defined as in section 3.2 in Boyce \cite{boyce}, $t0$ an arbitrary conveniently chosen point  and $y(t)$, the independent solutions solutions for the homogeneous differential equation:

\begin{equation}
	\label{eq_diff_homo}
	\ddot{y}(t) + \frac{\omega}{Q} \dot{y}(t) + \omega^2 y(t) = 0
\end{equation}

We first solve $y(t)$ by seeing that, for constant values of $\omega$,$m$ and $Q$, a solution for equation \ref{eq_diff_homo} is  $y(t) = e^(r \; t)$. Appliyng to equation \ref{eq_diff_homo} yields:

\begin{equation}
	\label{eq_characteristic}
	(r^2 + \frac{\omega}{Q} r + \omega^2) y(t) = 0
\end{equation}

\begin{equation*}
	r^2 + \frac{\omega}{Q} r + \omega^2 = 0
\end{equation*}

We find two solutions for r:

\begin{equation*}
	r_1 = \frac{-\frac{\omega}{Q} + \sqrt{\left( \frac{\omega}{Q} \right)^2 -4 \omega^2}}{2}
\end{equation*}

\begin{equation*}
	r_2 = \frac{-\frac{\omega}{Q} + \sqrt{\left( \frac{\omega}{Q} \right)^2 -4 \omega^2}}{2}
\end{equation*}

We use this to define:

\begin{equation*}
y_1(t) = e^{r_1 t}
\end{equation*}
\begin{equation*}
y_2(t) = e^{r_2 t}
\end{equation*}

If we solve equation \ref{eq_particular} for the obtained values of $y_1(t)$ and $y_2(t)$, we get the following:

\begin{equation}
	
\end{equation}

With $c_1$ and $c_2$ two constant variables that depend on the boundary values. \\
Using the definition of 
For solving $x$, we will use the method of variation of parameters. Therefore, $c_1$ and $c_2$ are changed for time dependent functions $u_1(t)$ and $u_2(t)$;

\begin{equation*}
	x = u_1(t) \: y_1(t) + u_2(t) \: y_2(t)
\end{equation*}

Differentiating with respect to $t$ yields:

\begin{equation*}
	\dot{x}  = u_1(t) \: r_1 \: y_1(t) + u_2(t) \: r_2 \: y_2(t) + \dot{u}_1(t) \: y_1(t) + \dot{u}_2(t) \: y_2(t)
\end{equation*}

Since there are two unknown functions $u_1(t)$ and $u_2(t)$ and only one equation to solve those. We may impose an extra condition on $u_1(t)$ and $u_2(t)$. If we require that:

\begin{equation}
	\label{eq_condition}
	\dot{u}_1(t) \: y_1(t) + \dot{u}_2(t) \: y_2(t) = 0 
\end{equation}

The equation for $\dot{x}$ reduces to:

\begin{equation*}
	\dot{x}  = u_1(t) \: r_1 \: y_1(t) + u_2(t) \: r_2 \: y_2(t) 
\end{equation*}

The second derivative of $x$ with respect to $t$ reads:

\begin{equation*}
	\ddot{x}  = u_1(t) \: r_1^2 \: y_1(t) + u_2(t) \: r_2^2 \: y_2(t) + \dot{u}_1(t) \: r_1 \: y_1(t) + \dot{u}_2(t) \: r_2 \: y_2(t)
\end{equation*}

If we now use these results in equation \ref{eq_diff} we get the following:

\begin{equation*}
	m \big[ u_1(t) \: r_1^2 \: y_1(t) + u_2(t) \: r_2^2 \: y_2(t) + \dot{u}_1(t) \: r_1 \: y_1(t) + \dot{u}_2(t) \: r_2 \: y_2(t) \big] + \frac{\omega \: m}{Q} \big[ u_1(t) \: r_1 \: y_1(t) + u_2(t) \: r_2 \: y_2(t) \big] + m \: \omega^2 \big[u_1(t) \: y_1(t) + u_2(t) \: y_2(t) \big]
\end{equation*}
	
\begin{equation*}
	u_1(t) \: m \big( \: r_1^2 + \frac{\omega \: m}{Q} r_1 +\omega^2 \big) y_1(t) + u_2(t) \: m \big( \: r_2^2 + \frac{\omega \: m}{Q} r_2 +\omega^2 \big) y_2(t) + \dot{u}_1(t) \: r_1 \: y_1(t) + \dot{u}_2(t) \: r_2 \: y_1(t) = F(t)
\end{equation*}

Using equation \ref{eq_characteristic} reduces the latter equation to:

\begin{equation*}
	 \dot{u}_1(t) \: r_1 \: y_1(t) + \dot{u}_2(t) \: r_2 \: y_1(t) = F(t)
\end{equation*}

We can solve this using equation \ref{eq_condition}:

\begin{equation*}
	\dot{u}_2(t) =  - \dot{u}_1(t) \: \frac{y_1(t)}{y_2(t)} 
\end{equation*}

Therefore we find:

\begin{equation*}
	\dot{u}_1(t) = F(t)
\end{equation*}






